%me=0 student solutions, me=1 - my solutions, me=2 - assignment
\def\me{0}
\def\num{2}  %homework number
\def\due{Wednesday, September 17}  %due date
\def\course{CSCI-GA.1170-001/002 Fundamental Algorithms} %course name
\def\name{Sahil Goel}
%
\iffalse
INSTRUCTIONS: replace # by the homework number.
(if this is not ps#.tex, use the right file name)

  Clip out the ********* INSERT HERE ********* bits below and insert
appropriate TeX code.  Once you are done with your file, run

  ``latex ps#.tex''

from a UNIX prompt.  If your LaTeX code is clean, the latex will exit
back to a prompt.  To see intermediate results, type

  ``xdvi ps#.dvi'' (from UNIX prompt)
  ``yap ps#.dvi'' (if using MikTex in Windows)

after compilation. Once you are done, run

  ``dvips ps#.dvi''

which should print your file to the nearest printer.  There will be
residual files called ps#.log, ps#.aux, and ps#.dvi.  All these can be
deleted, but do not delete ps1.tex. To generate postscript file ps#.ps,
run

  ``dvips -o ps#.ps ps#.dvi''

I assume you know how to print .ps files (``lpr -Pprinter ps#.ps'')
\fi
%
\documentclass[11pt]{article}
\usepackage{amsfonts}
\usepackage{latexsym}
\usepackage{qtree}
\setlength{\oddsidemargin}{.0in}
\setlength{\evensidemargin}{.0in}
\setlength{\textwidth}{6.5in}
\setlength{\topmargin}{-0.4in}
\setlength{\textheight}{8.5in}

\newcommand{\handout}[5]{
   \renewcommand{\thepage}{#1, Page \arabic{page}}
   \noindent
   \begin{center}
   \framebox{
      \vbox{
    \hbox to 5.78in { {\bf \course} \hfill #2 }
       \vspace{4mm}
       \hbox to 5.78in { {\Large \hfill #5  \hfill} }
       \vspace{2mm}
       \hbox to 5.78in { {\it #3 \hfill #4} }
      }
   }
   \end{center}
   \vspace*{4mm}
}

\newcounter{pppp}
\newcommand{\prob}{\arabic{pppp}}  %problem number
\newcommand{\increase}{\addtocounter{pppp}{1}}  %problem number

%first argument desription, second number of points
\newcommand{\newproblem}[2]{
\ifnum\me=0
\ifnum\prob>0 \newpage \fi
\increase
\setcounter{page}{1}
\handout{\name, Homework \num, Problem \arabic{pppp}}{\today}{Name: \name}{Due:
\due}{Solutions to Problem \prob\ of Homework \num\ (#2)}
\else
\increase
\section*{Problem \num-\prob~(#1) \hfill {#2}}
\fi
}

%\newcommand{\newproblem}[2]{\increase
%\section*{Problem \num-\prob~(#1) \hfill {#2}}
%}

\def\squarebox#1{\hbox to #1{\hfill\vbox to #1{\vfill}}}
\def\qed{\hspace*{\fill}
        \vbox{\hrule\hbox{\vrule\squarebox{.667em}\vrule}\hrule}}
\newenvironment{solution}{\begin{trivlist}\item[]{\bf Solution:}}
                      {\qed \end{trivlist}}
\newenvironment{solsketch}{\begin{trivlist}\item[]{\bf Solution Sketch:}}
                      {\qed \end{trivlist}}
\newenvironment{code}{\begin{tabbing}
12345\=12345\=12345\=12345\=12345\=12345\=12345\=12345\= \kill }
{\end{tabbing}}

\newcommand{\eqref}[1]{Equation~(\ref{eq:#1})}

\newcommand{\hint}[1]{({\bf Hint}: {#1})}
%Put more macros here, as needed.
\newcommand{\room}{\medskip\ni}
\newcommand{\brak}[1]{\langle #1 \rangle}
\newcommand{\bit}[1]{\{0,1\}^{#1}}
\newcommand{\zo}{\{0,1\}}
\newcommand{\C}{{\cal C}}

\newcommand{\nin}{\not\in}
\newcommand{\set}[1]{\{#1\}}
\renewcommand{\ni}{\noindent}
\renewcommand{\gets}{\leftarrow}
\renewcommand{\to}{\rightarrow}
\newcommand{\assign}{:=}

\newcommand{\AND}{\wedge}
\newcommand{\OR}{\vee}

\newcommand{\For}{\mbox{\bf For }}
\newcommand{\To}{\mbox{\bf to }}
\newcommand{\Do}{\mbox{\bf Do }}
\newcommand{\If}{\mbox{\bf If }}
\newcommand{\Then}{\mbox{\bf Then }}
\newcommand{\Else}{\mbox{\bf Else }}
\newcommand{\While}{\mbox{\bf While }}
\newcommand{\Repeat}{\mbox{\bf Repeat }}
\newcommand{\Until}{\mbox{\bf Until }}
\newcommand{\Return}{\mbox{\bf Return }}

\begin{document}

\ifnum\me=0
%\handout{PS\num}{\today}{Name: **** INSERT YOU NAME HERE ****}{Due:
%\due}{Solutions to Problem Set \num}
%
%I collaborated with *********** INSERT COLLABORATORS HERE (INDICATING
%SPECIFIC PROBLEMS) *************.
\fi
\ifnum\me=1
\handout{PS\num}{\today}{Name: Yevgeniy Dodis}{Due: \due}{Solution
{\em Sketches} to Problem Set \num}
\fi
\ifnum\me=2
\handout{PS\num}{\today}{Lecturer: Yevgeniy Dodis}{Due: \due}{Problem
Set \num}
\fi

\newproblem{Mergesort}{10  points}

\begin{itemize}

\item[(a)] (6 points)
Suppose you have some procedure {\sc FASTMERGE} that given two sorted lists of length $m$ each, merges them into one sorted list using $m^c$ steps for some constant $c >0$. Write a recursive algorithm using FASTMERGE to sort a list of
length $n$ and also calculate the run-time of this algorithm as a function of $c$. For what values of $c$ does the algorithm perform better than $O(n \log n)$.

\ifnum\me<2
\begin{solution}

MergeSort(A,low,high)
\{\\
if($low>=high$)\\
return;\\
$pivot=\frac{low+high}{2};$\\
MergeSort(A,low,pivot);\\
MergeSort(A,pivot+1,high);\\
FastMerge(A,low,pivot+1,high);\\
\}\\
\\

Now calculating the complexity of solution
$$T(n)=2*T(n/2) + (n/2)^c;$$
Using master's theorem, $f(n)=n^{ \log_b{a}}$\\
Therefore $f(n)=n^{ \log_2 (2)}$\\
f(n)=n\\
g(n)=$(n/2)^c$;
Now for the runtime algorithm to be perform better than $O(n \log n), f(n) > g(n)$\\
Therefore $n>(n/2)^c$\\
Taking log on both sides\\
$$ \log n > c \log (n/2) $$
$$c < \frac{\log n}{\log n/2}$$
$$ c < \frac{\log n}{\log n - \log 2}$$
For very large n
$$ c < 1 $$
if $c<1$, then complexity of function will be O(n)
\end{solution}
\fi

\item[(b)] (4 points) Let $A[1 \ldots n]$ be an array such that the first $n - \sqrt{n}$ elements are already in sorted order. Write an algorithm that will sort $A$ in substantially better than
$O(n \log n)$ steps.

\ifnum\me<2
\begin{solution}
Using the MergeSort from last question and Merge function to merge 2 sorted array in one complete sorted array\\
Step 1: Call MergeSort for last $\sqrt{n}$ elements\\
Step 2: Call Merge for 2 sorted arrays A1[1:n-$\sqrt{n}$] and A2[n-$\sqrt{n}$+1:n]\\

SortLast(A,n)\\
\{\\
MergeSort(A,$n-\sqrt{n}+1$,n);  //Calling Merge sort only for last $\sqrt{n}$ elements \\ 
Merge(A,0,$n-\sqrt{n}+1$,n); //Merging 2 sorted arrays \\
\}\\
\\
MergeSort(A,low,high)
\{\\
if($low>=high$)\\
return;\\
$pivot=\frac{low+high}{2};$\\
MergeSort(A,low,pivot);\\
MergeSort(A,pivot+1,high);\\
FastMerge(A,low,pivot+1,high);\\
\}\\
\\
In merge function, l is the first index for first sorted array and m is the first index for second sorted array and h is the last element in second array\\
Merge(A,l,m,h)\\
\{\\
int b1[m-l],b2[h-m+1],i=0,j=0,lower=l;\\
//storing two sorted arrays in temporary arrays b1 and b2\\
    while$(i<m-l)$\\
    \{\\
        b1[i]=a[l+i];\\
        i++;\\
    \}\\
    while$(j<h-m+1)$\\
    \{\\
        b2[j]=a[m+j];\\
        j++;\\
    \}\\
  i=0;\\
  j=0;\\
  //Now merging two arrays and comparing only b1[i] and b2[j] \\
    while$(i<m-lower \&\& j<h-m+1)$\\
    \{\\
        if$(b1[i]<b2[j])$\\
            a[l++]=b1[i++];\\
        else\\
            a[l++]=b2[j++];\\
    \}\\
    // b2 has reached end therefore copy all the elements from b1\\
    while$(i<m-lower)$\\
    \{\\
        a[l++]=b1[i++];\\   
    \}\\
        // b1 has reached end therefore copy all the elements from b2\\
    while$(j<h-m+1)$\\
    \{\\
        a[l++]=b2[j++];\\
    \}\\
\}\\

Total time complexity of code will be T(n), where
$$T(n)= T(MergeSort) + T(Merge)$$
To MergeSort last $\sqrt{n}$ elements, time complexity is O$(\sqrt{n} * \log\sqrt{n})$
And to Merge n elements, time complexity is O(n)\\
Therefor total complexity is  O$(\sqrt{n} * \log\sqrt{n})$ + O(n)\\
Since $\log\sqrt{n} < \sqrt{n}$\\
Multiplying with $\sqrt{n}$ on both sides\\
$\sqrt{n} * \log\sqrt{n} < \sqrt{n}*\sqrt{n}$  \\
Therefore for very large n, O(n) will dominate and time complexity will be O(n)
\end{solution}
\fi

\end{itemize}

\newproblem{Functionality vs. Running Time}{10 points}

Consider the following recursive procedure.

\medskip

\begin{code}
\>{\sc Bla}$(n)$:\\
\> \> \If $n=1$ \Then \Return $1$\\
\> \> \Else \Return $\mbox{\sc Bla}(n/2) + \mbox{\sc Bla}(n/2) +
\mbox{\sc Bla}(n/2)$
\end{code}

\begin{itemize}

\item[(a)] (3 points) What function of $n$ does {\sc Bla} compute (assume it is
always called on $n$ which is a power of $2$)?

\ifnum\me<2
\begin{solution}
At every step, the result is multiplied by 3 and it is recursively called $\log_2{n}$ times\\
Therefore the function computes $3^{\log_2{n}}$
\end{solution}
\fi

\item[(b)] (3 points) What is the running time $T(n)$ of {\sc Bla}?

\ifnum\me<2
\begin{solution}
$$T(n)=T(n/2) + T(n/2) + T(n/2)$$
$$T(n)=3*T(n/2)$$
Replacing n with $2^k$
$$T(2^k)=3*T(2^{k-1})$$
Replacing $T(2^k) with S(k)$\\
$$S(k)=3*S(k-1)$$
Dividing by $3^k$\\
$$S(k)/3^k = S(k-1)/3^{k-1}$$
Assume $S(k)/3^k = P(k)$
$$P(k)=P(k-1)$$
Therefore $$P(k)=P(0)=1$$
$$P(k)=1$$
$$S(k)=3^k * 1$$
$$T(2^k)=3^k$$
Since $k=\log_2 n$
$$T(n)=3^{\log_2 n}$$
$$T(n)=n^{\log_2 3}$$
$$T(n)=n^{1.58}$$

\end{solution}
\fi

\item[(c)] (4 points) How do the answers to (a) and (b) change if we replace the
last line by\\ ``\Else \Return $3\cdot \mbox{\sc Bla}(n/2)$''?

\ifnum\me<2
\begin{solution}
(a) part remains same but answer to (b) part changes\\
Now BLA(n/2) is only called one time as compared to three times in previous step.\\
Assuming multiplication with 3 takes c time
$$T(n)=T(n/2) + c$$ 
Substituting $n=2^k$
$$T(2^k)=T(2^{k-1}) + c$$
$$S(k)=S(k-1) + c$$
$$S(k)=S(k-2) + c + c$$
Similarly
$$S(k)=c+c+c+\ldots+c (k times) $$
$$S(k)=kc$$
$$T(2^k)=kc$$
Substituting $k=\log_2 {n}$
$$T(n)=\log_2 {n} * c$$
Therefore total complexity is $O(\log_2 {n})$


\end{solution}
\fi

\end{itemize}

\newproblem{Different Methods for Recurrences}{14 points}

Consider the recurrence $T(n) = 8T(n/4) + n$ with initial condition
$T(1)=1$.

\begin{itemize}

\item[(a)] (2 points) Solve it asymptotically using the ``master theorem''.

\ifnum\me<2
\begin{solution}
$$f(n)=n^{\log_b a}$$
$$f(n)=n^{\log_4 8}$$
$$f(n)=n^{3/2{\log_2 2}}$$
$$f(n)=n^{\frac{3}{2}}$$
$$g(n)=n^{\log_b a - \epsilon}$$
$$g(n)=n^{3/2 - \epsilon}$$
Since g(n)=n and $\epsilon=1/2$ 
$$T(n)=O(f(n))$$
$$T(n)=O(n^{\frac{3}{2}})$$
\end{solution}
\fi

\item[(b)] (4 points) Solve it by the ``guess-then-verify method''. Namely, guess a
function $g(n)$ --- presumably solving part (a) will give you a good
guess --- and argue by induction that for all values of $n$ we have
$T(n) \le g(n)$. What is the ``smallest'' $g(n)$ for which your
inductive proof works?

\ifnum\me<2
\begin{solution}
This solution has 2 parts, one will be wrong guess because of loose bounds and other will be right guess because of tight bounds.\\
Case 1:\\
Let us assume T(n)=O(g(n))\\
and We guess $g(n)=n^{3/2}$\\
Therefore the condition that should hold is \\
There $\exists c>0$  s.t. $T(n) \leq c*(n^{3/2})$\\
For n=1
$$1=T(1)\stackrel{}{\leq}c*(1)^{3/2}$$
Therefore  $c {\geq} 1$\\
Assuming the solution is true for 1,2,\ldots,n-1 , we have to prove it is true for n where $n>1$
$$T(n)=8*T(n/4) + n $$
Since $T(n) \leq c*(n^{3/2})$
$$8*(c*(n/4)^{3/2}) + n \leq c*(n^{3/2})$$ 
$$ \frac{8*c*n^{3/2}}{4^{3/2}} + n \leq c*n^{3/2}$$
$$ c*n^{3/2} + n \leq c*n^{3/2}$$
Therefore c term gets cancelled out, so the proof goes wrong\\
\\
Case II\\
Let us guess $g(n)=n^{3/2} - n$ and T(n)=O(g(n)\\
Therefore the condition that should hold is \\
There $\exists c>0$  s.t. $T(n) \leq c1*(n^{3/2}) - c2*n$\\
For n=1
$$1=T(1)\stackrel{}{\leq}c1*(1)^{3/2} - c2$$
$$ c1 \geq 1+c2$$
Assuming the solution is true for 1,2,\ldots,n-1 , we have to prove it is true for n where $n>1$
$$T(n)=8*T(n/4) + n $$
Since $T(n) \leq c1*(n^{3/2}) - c2*(n) $
$$ 8*[c1*(n/4)^{3/2} - c2 * (n/4)] + n \leq c1*n^{3/2} - c2*n$$
$$ c1*n^{3/2} - 2*c2*n + n \leq c1*n^{3/2} - c2*n$$
$$ -2*c2 + 1 \leq -c2$$
$$ 1 \leq c2$$
and Since $c1 \geq 1 + c2$\\
$$c1 \geq 2 $$
$$c2 \geq 1 $$
The minimum value of g(n) will be $2*n^{3/2}  - n$
\end{solution}
\fi

\item[(c)] (4 points) Solve it by the ``recursive tree method''. Namely, draw the
full recursive tree for this recurrence, and sum up all the value to
get the final time estimate. Again, try to be as precise as you can
(i.e., asymptotic answer is OK, but would be nice if you preserve a
``leading constant'' as well).

\ifnum\me<2
\begin{solution}
[.NP {\bf b} c ] 

\Tree [.T(n) 8T(n/4) n ]\\
Expanding tree to second step\\
\Tree [.T(n) [.8T(n/4) 64T(n/4) 2n ] n ]\\
\\
Expanding more\\
\\
\Tree [.T(n) [.8T(n/4) [.64(T(n/16)) 512(T(n/64)) 4n ]  2n ] n ]\\
\\
\\
Similarly expanding and summing all the value\\
\\
$$n + 2n + 4n + \ldots + 2^{log_4 n}n $$
$$\Longrightarrow n + 2n + 4n + \ldots + \sqrt{n}*n $$
$$\Longrightarrow n(1+ 2 + 4 + 8 + \ldots + \sqrt{n}*n)$$
$$\Longrightarrow n(2*\sqrt{n} - 1)$$
$$\Longrightarrow 2*n^{3/2} - n$$
Hence $T(n) = 2*n^{3/2} - n = O(n^{3/2})$
 




\end{solution}
\fi

\item[(d)] (4 points) Solve it {\em precisely} using the ``domain-range
substitution'' technique. Namely, make several changes of variables
until you get a basic recurrence of the form $S(k) = S(k-1) + f(k)$
for some $f$, and then compute the answer from there. Make sure you
carefully maintain the correct initial condition.

\ifnum\me<2
\begin{solution}
$$T(n)=8*T(n/4) + n $$
Substituting $n=4^k$\\
$$T(4^k) = 8*T(4^{k-1}) + 4^k$$
Lets $T(4^k) = S(k)$
$$S(k)=8*S(k-1) + 4^k$$
Dividing whole equation by $8^k$
$$\frac{S(k)}{8^k} = \frac{S(k-1)}{8^{k-1}} + \frac{1}{2^k}$$
Now lets $P(k)=S(k)/8^k$
$$P(k)=P(k-1) +  \frac{1}{2^k}$$
$$P(k)=P(k-2) +  \frac{1}{2^{k-1}} +  \frac{1}{2^k}$$
$$P(k)=  \frac{1}{2^k} +  \frac{1}{2^{k-1}} + \ldots +  \frac{1}{2} +  \frac{1}{2^0}$$
$$P(k)= 2*[1 - \frac{1}{2^{k+1}}]$$
$$P(k)= 2 - \frac{1}{2^{k}}$$
Now $S(k)=P(k)*8^k$
$$S(k)=2*8^k - \frac{8^k}{2^k}$$
$$T(4^k)=2*8^k - \frac{8^k}{2^k}$$
Now $k=\log_4 n$
$$T(n)=2*n^{3/2} - n$$

\end{solution}
\fi

\item[(e)] This part will not be graded. However, briefly describe
your personal comparison of the above 4 methods. Which one was the
fastest? The easiest? The most precise?

\ifnum\me<2
\begin{solution}
The fastest one is master's theorem because we just have to check the values of function f(n) and g(n) which is very easy to calculate and just compare the 2 functions\\
The easiest is also I think master's theorem if we don't consider the exact preciseness\\
The Most precise method is substitution method because we are able to find even the constant factors\\
The order according to me\\
For easiness, Most easiest one first  \\
$Master's Theorem > Substitution Method > Recursive Tee > Guess and verify$\\
For fastness, fastest one first  \\
$Master's Theorem > Recursive Tree > Substitution Method  > Guess and verify$\\
For Preciseness, most precise one first  \\
$Substitution method > Recursive Tree > Master's Theorem > Guess and Verify$

\end{solution}
\fi

\end{itemize}

\newproblem{More Recurrences}{12 points}

Solve the following recurrences using any method you like. If you use
``master theorem'', use the version from the book and justify why it
applies. Assume $T(1) = 2$, and be sure you explain every important
step.

\begin{itemize}

\item[(a)] (3 points) $T(n)  = T(2n/3) + \sqrt{n}$.

\ifnum\me<2
\begin{solution}
$$T(n)  = T(2n/3) + \sqrt{n}$$\\
Substituting $n=(3/2)^k$
$$T({(3/2)}^k) = T({(3/2)}^{k-1}) + (3/2)^{k/2}$$
Substituting $T((3/2)^k)=S(k)$
$$S(k)=S(k-1)+(3/2)^{k/2}$$
$$S(k)=S(k-2)+(3/2)^{(k-1)/2}+(3/2)^{k/2}$$
$$S(k)=2+(3/2)^{1/2}+\ldots+(3/2)^{(k-1)/2}+(3/2)^{k/2}$$
$$S(k)=1+1+(3/2)^{1/2}+\ldots+(3/2)^{(k-1)/2}+(3/2)^{k/2}$$

$$S(k)=1+\frac{[(3/2)^{(k+1)/2}]-1}{(3/2)^{1/2} - 1}$$
$$S(k)=1+\frac{[(3/2)^{1/2}(3/2)^{k/2}]-1}{(3/2)^{1/2} - 1}$$
Substituting $k=\log_{3/2} n$\\
$$T(n)=1+\frac{[(3/2)^{1/2}(3/2)^{(\log_{3/2} n)/2}]-1}{(3/2)^{1/2} - 1}$$
$$T(n)=1+\frac{(3/2)^{1/2}*(n)^{(1/2)}-1}{(3/2)^{1/2}-1}$$
$$T(n)=9.169\sqrt{n} - 7.169$$
Therefore \\
$T(n) = \Theta(\sqrt{n})$
\end{solution}
\fi

\item[(b)] (4 points) $T(n)  = T(n/2) + \log n$.

\ifnum\me<2
\begin{solution}
$$T(n)  = T(n/2) + \log n$$
Substituting $n=2^k$
$$T(2^k)=T(2^{k-1}) + \log 2^k$$
$$T(2^k)=T(2^{k-1}) + k\log 2$$
$$T(2^k)=T(2^{k-1}) + k$$
lets $S(k)=T(2^k)$\\
$$S(k)=S(k-1) + k$$
$$S(k)=S(k-2)+ k-1 + k$$
$$S(k)=S(0) + 1 + 2 + \ldots + k-1 + k$$
Since S(0)=T(1)\\
$$S(k)= 2 + \frac{k*(k+1)}{2}$$
$$S(k)= \frac{4+k^2+k}{2}$$
Substituting k=log n\\
$$T(n)=\frac{4 + (\log n)^2 + (\log n)}{2}$$
Therefore $T(n)=\Theta((\log n)^2)$


\end{solution}
\fi

\item[(c)] (5 points) $T(n)  = T(\sqrt{n}) + 1$.
\hint{Substitute $\ldots$ until you are done!}

\ifnum\me<2
\begin{solution}
$$T(n)  = T(\sqrt{n}) + 1$$
Substituting $n=2^k$
$$T(2^k) = T(2^{k/2}) + 1$$
Substituting $S(k)=T(2^k)$\\
$$S(k)=S(k/2) + 1$$
Substituting $m=2^k$
$$S(2^m) = S(2^{m-1}) + 1 $$
Lets $P(m)=S(2^m)$\\
$$P(m)=P(m-1)+1$$
$$P(m)=P(m-2)+1+1$$
$$P(m)=1+1+\ldots+1+P(1)$$
$$P(m)=m-1+P(1)$$

Since $T(2)=T(1.41)+1 \Longrightarrow T(2)=T(1)+1 \Longrightarrow T(2)=3$\\

$$P(m)=1+1+\ldots+1 + 3$$

$$P(m)=m+2$$
Since $m=log k$
$$S(k)=\log k + 2$$
Since $k=\log n$
$$T(n)=\log ({\log n}) + 2$$
Therefore $T(n)=\Theta(\log {\log n})$

\end{solution}
\fi

\end{itemize}

\end{document}
