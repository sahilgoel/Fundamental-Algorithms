%me=0 student solutions, me=1 - my solutions, me=2 - assignment
\def\me{0}
\def\num{1}  %homework number
\def\due{Tuesday, September 10}  %due date
\def\course{CSCI-GA.1170-001/002 Fundamental Algorithms} %course name
\def\name{Sahil Goel}
%
\iffalse
INSTRUCTIONS: replace # by the homework number.
(if this is not ps#.tex, use the right file name)

  Clip out the ********* INSERT HERE ********* bits below and insert
appropriate TeX code.  Once you are done with your file, run

  ``latex ps#.tex''

from a UNIX prompt.  If your LaTeX code is clean, the latex will exit
back to a prompt.  To see intermediate results, type

  ``xdvi ps#.dvi'' (from UNIX prompt)
  ``yap ps#.dvi'' (if using MikTex in Windows)

after compilation. Once you are done, run

  ``dvips ps#.dvi''

which should print your file to the nearest printer.  There will be
residual files called ps#.log, ps#.aux, and ps#.dvi.  All these can be
deleted, but do not delete ps1.tex. To generate postscript file ps#.ps,
run

  ``dvips -o ps#.ps ps#.dvi''

I assume you know how to print .ps files (``lpr -Pprinter ps#.ps'')
\fi
%
\documentclass[11pt]{article}
\usepackage{amsfonts}
\usepackage{latexsym}
\setlength{\oddsidemargin}{.0in}
\setlength{\evensidemargin}{.0in}
\setlength{\textwidth}{6.5in}
\setlength{\topmargin}{-0.4in}
\setlength{\textheight}{8.5in}

\newcommand{\handout}[5]{
   \renewcommand{\thepage}{#1, Page \arabic{page}}
   \noindent
   \begin{center}
   \framebox{
      \vbox{
    \hbox to 5.78in { {\bf \course} \hfill #2 }
       \vspace{4mm}
       \hbox to 5.78in { {\Large \hfill #5  \hfill} }
       \vspace{2mm}
       \hbox to 5.78in { {\it #3 \hfill #4} }
      }
   }
   \end{center}
   \vspace*{4mm}
}

\newcounter{pppp}
\newcommand{\prob}{\arabic{pppp}}  %problem number
\newcommand{\increase}{\addtocounter{pppp}{1}}  %problem number

%first argument desription, second number of points
\newcommand{\newproblem}[2]{
\ifnum\me=0
\ifnum\prob>0 \newpage \fi
\increase
\setcounter{page}{1}
\handout{\name, Homework \num, Problem \arabic{pppp}}{\today}{Name: \name}{Due:
\due}{Solutions to Problem \prob\ of Homework \num\ (#2)}
\else
\increase
\section*{Problem \num-\prob~(#1) \hfill {#2}}
\fi
}

%\newcommand{\newproblem}[2]{\increase
%\section*{Problem \num-\prob~(#1) \hfill {#2}}
%}

\def\squarebox#1{\hbox to #1{\hfill\vbox to #1{\vfill}}}
\def\qed{\hspace*{\fill}
        \vbox{\hrule\hbox{\vrule\squarebox{.667em}\vrule}\hrule}}
\newenvironment{solution}{\begin{trivlist}\item[]{\bf Solution:}}
                      {\qed \end{trivlist}}
\newenvironment{solsketch}{\begin{trivlist}\item[]{\bf Solution Sketch:}}
                      {\qed \end{trivlist}}
\newenvironment{code}{\begin{tabbing}
12345\=12345\=12345\=12345\=12345\=12345\=12345\=12345\= \kill }
{\end{tabbing}}

\newcommand{\eqref}[1]{Equation~(\ref{eq:#1})}

\newcommand{\hint}[1]{({\bf Hint}: {#1})}
%Put more macros here, as needed.
\newcommand{\room}{\medskip\ni}
\newcommand{\brak}[1]{\langle #1 \rangle}
\newcommand{\bit}[1]{\{0,1\}^{#1}}
\newcommand{\zo}{\{0,1\}}
\newcommand{\C}{{\cal C}}

\newcommand{\nin}{\not\in}
\newcommand{\set}[1]{\{#1\}}
\renewcommand{\ni}{\noindent}
\renewcommand{\gets}{\leftarrow}
\renewcommand{\to}{\rightarrow}
\newcommand{\assign}{:=}

\newcommand{\AND}{\wedge}
\newcommand{\OR}{\vee}

\newcommand{\For}{\mbox{\bf For }}
\newcommand{\To}{\mbox{\bf to }}
\newcommand{\Do}{\mbox{\bf Do }}
\newcommand{\If}{\mbox{\bf If }}
\newcommand{\Then}{\mbox{\bf Then }}
\newcommand{\Else}{\mbox{\bf Else }}
\newcommand{\While}{\mbox{\bf While }}
\newcommand{\Repeat}{\mbox{\bf Repeat }}
\newcommand{\Until}{\mbox{\bf Until }}
\newcommand{\Return}{\mbox{\bf Return }}

\begin{document}

\ifnum\me=0
%\handout{PS\num}{\today}{Name: **** INSERT YOU NAME HERE ****}{Due:
%\due}{Solutions to Problem Set \num}
%
%I collaborated with *********** INSERT COLLABORATORS HERE (INDICATING
%SPECIFIC PROBLEMS) *************.
\fi
\ifnum\me=1
\handout{PS\num}{\today}{Name: Yevgeniy Dodis}{Due: \due}{Solution
{\em Sketches} to Problem Set \num}
\fi
\ifnum\me=2
\handout{PS\num}{\today}{Lecturer: Yevgeniy Dodis}{Due: \due}{Problem
Set \num}
\fi


\newproblem{Polynomial Evaluation}{16 (+4) points}

A degree-$n$ polynomial $P(x)$ is a function
$$P(x) = a_0 + a_1x + \ldots + a_{n-1}x^{n-1} + a_n x^n = \sum_{i=0}^n a_i
x^i$$

\begin{itemize}

\item[(a)] (2 points) Express the value $P(x)$ as
$$P(x) = a_0 + a_1x + \ldots + a_{n-2} x^{n-2} + b_{n-1} x^{n-1} =
\sum_{i=0}^{n-1} b_i x^i$$
where $b_0=a_0,\ldots, b_{n-2} = a_{n-2}$. What is $b_{n-1}$ as a
function of the $a_i$'s and $x$?

\ifnum\me<2
\begin{solution}

$$P(x) = a_0 + a_1x + \ldots + a_{n-1}x^{n-1} + a_n x^n = \sum_{i=0}^n a_i
x^i                      - 1$$
$$P(x) = a_0 + a_1x + \ldots + a_{n-2} x^{n-2} + b_{n-1} x^{n-1} =
\sum_{i=0}^{n-1} b_i x^i                   -2 $$
Equating equations 1 and 2, since $a_0=b_0, a_1=b_1, ...., a_{n-2} = b_{n-2}$
$$a_{n-1}*x^{n-1} + a_n*x^n = b_{n-1}*x^{n-1}$$
$$b_{n-1} = a_{n-1} + a_n*x$$
$$P(x) = b_0 + b_1x + \ldots + b_{n-2} x^{n-2} + b_{n-1} x^{n-1} =
\sum_{i=0}^{n-1} b_i x^i  $$
where $a_0=b_0, a_1=b_1, . . ., a_{n-2}=b_{n-2}$ and $b_{n-1}=a_{n-1} + a_n*x$

\end{solution}
\fi

\item[(b)] (5 points) Using part (a) above write a recursive procedure
$\mathsf{Eval}(A,n,x)$ to evaluate the polynomial $P(x)$ whose
coefficients are given in the array $A[0\ldots n]$ (i.e., $A[0]=a_0$,
etc.). Make sure you do not forget the base case $n=0$.

\ifnum\me<2
\begin{solution}
There are 3 solutions to this problem.\\
1. With base case n=0\\
2. With base case n=-1\\
3. With base case n=degree of polynomial (Assumption : if degree of polynomial is accessible inside the function and initially calling the function as eval(A,x,0)). This is the most optimal solution amongst the three\\

1. Base case n=0

Eval(A,x,n)\\
\{\\
if(n==0)\\
return A[0];\\
return $x^n*A[n]$ + Eval(A,x,n-1);\\
\}\\

2. Base case n=-1

Eval(A,x,n)\\
\{\\
if(n==-1)\\
return 0;\\
return $x^n*A[n]$ + Eval(A,x,n-1);\\
\}\\

3. Base case n=degree of polynomial\\
    Call from main(), Eval(A,x,0)\\
    Assumption: Degree of polynomial known inside function\\
    Horner's Rule using recursion \\
    
    Eval(A,x,n)\\
\{\\
if(n==DegreeOfPolynomial)\\
return A[n];\\
return $A[n] + x*Eval(A,x,n+1);$\\
\}\\

    
    

\end{solution}
\fi

\item[(c)] (3 points) Let $T(n)$ be the running time of your implementation
of $\mathsf{Eval}$. Write a recurrence equation for $T(n)$ and solve
it in the $\Theta(\cdot)$ notation.

\ifnum\me<2
\begin{solution}
Assuming calculation of $x^n$ takes $O(n)$ times\\
For the first 2 solutions of 1-(c), the recurrence equation will become\\
$$ T(1)=c_1$$
$$ T(n)= T(n-1) + c*n $$
Solving the above recurrence equation\\
$$T(n)= T(n-2) + c*(n-1) + c*n$$
$$T(n)= T(n-3) + c*(n-2) + c*(n-1) + c*n$$
Solving in the similar manner
$$T(n)= T(1) + c*(n+ (n-1) + (n-2) + \ldots + 1)$$
$$T(n)=c1 + c*((n)(n+1)/2)$$
Therefore $T(n) = O(n^2) $

Now calculating the complexity for horner's rule method (iii) Part\\

$$T(1)=c_1$$
$$T(n)=  T(n-1) + c$$
Solving for the recurrence equation above\\
$$T(n)=  T(n-2) + c+ c$$
Solving similarly \\
$$T(n) = n*c $$
Therefore $T(n)=O(n)$ which is much better than $O(n^2)$




\end{solution}
\fi

\item[(d)] (6 points) Assuming $n$ is a power of $2$, try to express $P(x)$ as
$P(x) = P_0(x) + x^{n/2}P_1(x)$, where $P_0(x)$ and $P_1(x)$ are both
polynomials of degree $n/2$. Assuming the computation of $x^{n/2}$
takes $O(n)$ times, describe (in words or pseudocode) a recursive
procedure $\mathsf{Eval}_2$ to compute $P(x)$ using two recursive
calls to $\mathsf{Eval}_2$. Write a recurrence relation for the
running time of $\mathsf{Eval}_2$ and solve it. How does your solution
compare to your solution in part (c)?

\ifnum\me<2
\begin{solution}
If n is power of 2 
$P(x)$ can be split into $P_0(x)$ and $P_1(x)$. For example suppose
$$P(x) =  a_0 + a_1*x+a_2*x^2+a_3*x^3+a_4*x^4$$  
It could be divided as 
$$P(x)=P_0(x) + x^2*P_1(x)$$
where $P_0(x) =   a_0 + a_1*x+a_2*x^2$ and $P_1(x) = a_3*x + a_4*x^2$
\\
Now both the polynomials are 2 degree polynomials and both can be recursively called to evaluate the value of expression\\

Eval2(A,x,l,h)  \hfill    \%Where l is lower bound and h is higher bound \\
\{\\
if(l==h)    \\
      return A[l]; \\
    return Eval2(A,x,l,((l+h)/2)) + $[x^{(h-l)/2+1}]$*Eval2( A, x, ((l+h)/2)+1 , h);\\
\}\\
\\
Here First call to Eval2 is for first half of polynomial i.e. from 0 to n/2 i.e. (l+h)/2. Second call to Eval2 is for second half of polynomial i.e. from (n/2)+1 i.e. ((l+h)/2 + 1) to h.
\\
Recurrence Equation will become:
$$T(1) = c_1$$
since calculation of $x^{n/2}$ takes  O(n) time, time consumed in that step is $c_2*n$ \\
$$T(n)= 2*T(n/2) + c_2*n;$$ 
By substitution \\
$$T(n)= 2*(2*T(n/4) + c_2*(n/2))+ c_2*n;$$ 
$$T(n)=4*T(n/4) + 2*c_2*n$$
Substituting in similar fashion\\
$$T(n)=n*T(1) + log(n)*n$$
$$T(n)=n*c+ log(n)*n$$
Therefore $T(n)=O(n*log(n))$

\end{solution}
\fi

\item[(e)] ({\bf Extra Credit.}) Explain how to fix the slow
``conquer'' step of part (d) so that the resulting solution is as
efficient as ``expected''.

\ifnum\me<2
\begin{solution}
Slow conquer step i.e. calculation of $x^n$ can be fixed by calculating $x^n$ in $log(n)$ time.\\
Algorithm for the same will be\\
POW(x,n)\\
\{
if(n==0)\\
return 1;\\
temp=POW(x,$n/2$);\\
if(n\%2==0);\\
return temp*temp;\\
else\\
return x*temp*temp;\\
\}\\

Total complexity of algorithm will become $O(n)$
\end{solution}
\fi

\end{itemize}

\newproblem{Asymptotic Comparisons}{10 Points}

For each of the following pairs of functions $f(n)$ and $g(n)$, state
whether $f$ is $O(g)$; whether $f$ is $o(g)$; whether $f$ is
$\Theta(g)$; whether $f$ is $\Omega(g)$; and whether $f$ is
$\omega(g)$.  (More than one of these can be true for a single pair!)

\begin{itemize}
\item[(a)] $f(n) = 32 n^{21} + 2$; $g(n) = \frac{n^{22}+3n+4}{111} - 52 n$.

\ifnum\me<2
\begin{solution}
$f$ is $o(g)$\\
$f$ is $O(g)$\\
since $\lim_{n\to\infty}\frac{f(n)}{g(n)}=0$\\

\end{solution}
\fi

\item[(b)]  $f(n) = \log(n^{21}+3n)$; $g(n) = \log(n^{2}-1)$.

\ifnum\me<2
\begin{solution}
$f$ is $\Omega(g)$\\
$f$ is $O(g)$\\
$f$ is $\Theta(g)$\\

\end{solution}
\fi

\item[(c)]  $f(n) = \log (2^{n}+n^2)$; $g(n) = \log(n^{22})$.

\ifnum\me<2
\begin{solution}
$f$ is $\omega(g)$\\
$f$ is $\Omega(g)$\\

\end{solution}
\fi

\item[(d)] $f(n) = n^{3} \cdot 2^{n}$; $g(n) = n^{2} \cdot 3^{n}$.

\ifnum\me<2
\begin{solution}
$f$ is $o(g)$\\
$f$ is $O(g)$\\
\end{solution}
\fi

\item[(e)] $f(n) = (n^{n})^{3}$; $g(n)=n^{(n^{3})}$.

\ifnum\me<2
\begin{solution}
$f$ is $o(g)$\\
$f$ is $O(g)$\\

\end{solution}
\fi

\end{itemize}

\newproblem{The Same or Not the Same?}{10 points}

The following two functions both take as arguments two $n$-element
arrays $A$ and $B$:

\begin{code}
{\sc Magic-1}$(A,B,n)$ \\
\>  \For $i = 1$ \To $n$\\
\>\>  \For $j = 1$ \To $n$\\
\>\>\>  \If $A[i] \ge B[j]$ \Return FALSE\\
\>  \Return TRUE\\
\end{code}

\begin{code}
{\sc Magic-2}$(A,B,n)$ \\
\>  $temp := A[1]$\\
\> \For $i = 2$ \To $n$\\
\>\>  \If $A[i] > temp$ \Then $temp := A[i]$\\
\> \For $j = 1$ \To $n$\\
\>\> \If $temp \ge B[j]$ \Return FALSE\\
\>  \Return TRUE\\
\end{code}


\begin{itemize}

\item[(a)] (2 points) It turns out both of these procedures return
  TRUE if and only if the same `special condition' regarding the
    arrays $A$ and $B$ holds. Describe this `special condition' in
  English.

\ifnum\me<2
\begin{solution}
If every value in the array $A$ is smaller than any of the values in array $B$ then the result is TRUE. In other words, if the maximum of array $A$ elements is smaller than the minimum of array $B$ elements, than the result will be true. 
\end{solution}
\fi

\item[(b)] (5 points) Analyze the worst-case running time for both
  algorithms in the $\Theta$-notation. Which algorithm would you
  chose? Is it the one with the shortest code (number of lines)?

\ifnum\me<2
\begin{solution}
In worst case, considering MAGIC-1 code,
the outer loop will run n times and the inner loop will run n times for every outer loop.\\
So total number of steps will be $n^2$\\
Hence the worst case complexity of algorithm is $\Theta(n^2)$

In worst case, considering Magic-2 Code,
Finding the maximum of array $A$ will take n comparisons i.e. if the element is at the end of array $A$.\\
In next step worst case will be when maximum of array $A$ is smaller than every element of $B$ except the last element. It will also take n comparisons.
\\
Total Complexity will become $\Theta(n)$.\\
More optimal solution is the second code with more lines of code, so it is not the one with shortest lines of code.\\
\end{solution}
\fi

\item[(c)] (3 points) Does the situation change if we consider the
  best-case running time for both algorithms?

\ifnum\me<2
\begin{solution}
Yes the situation changes if we consider the best case running time algorithm for both algorithms because in Magic-1, there will be only one comparison in best case that is when first element of A will be larger than the first element of array B, it will exit and return false.\\
Whereas in Magic-B, there will be minimum n comparisons to find out the maximum of array $A$.\\
Therefore, Magic-1 best case running time is better than Magic-2 best case running time.

\end{solution}
\fi

\end{itemize}

\end{document}
