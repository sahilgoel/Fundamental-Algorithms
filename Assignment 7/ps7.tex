%%me=0 student solutions, me=1 - my solutions, me=2 - assignment
\def\me{0}
\def\num{7}  %homework number
\def\due{Wednesday, November 5}  %due date
\def\course{CSCI-GA.1170-001/002 Fundamental Algorithms} %course name
\def\name{Sahil Goel}
%
\iffalse
INSTRUCTIONS: replace # by the homework number.
(if this is not ps#.tex, use the right file name)

  Clip out the ********* INSERT HERE ********* bits below and insert
appropriate TeX code.  Once you are done with your file, run

  ``latex ps#.tex''

from a UNIX prompt.  If your LaTeX code is clean, the latex will exit
back to a prompt.  To see intermediate results, type

  ``xdvi ps#.dvi'' (from UNIX prompt)
  ``yap ps#.dvi'' (if using MikTex in Windows)

after compilation. Once you are done, run

  ``dvips ps#.dvi''

which should print your file to the nearest printer.  There will be
residual files called ps#.log, ps#.aux, and ps#.dvi.  All these can be
deleted, but do not delete ps1.tex. To generate postscript file ps#.ps,
run

  ``dvips -o ps#.ps ps#.dvi''

I assume you know how to print .ps files (``lpr -Pprinter ps#.ps'')
\fi
%
\documentclass[11pt]{article}
\usepackage{amsfonts,amsmath}
\usepackage{latexsym}
\setlength{\oddsidemargin}{.0in}
\setlength{\evensidemargin}{.0in}
\setlength{\textwidth}{6.5in}
\setlength{\topmargin}{-0.4in}
\setlength{\textheight}{8.5in}

\newcommand{\handout}[5]{
   \renewcommand{\thepage}{#1, Page \arabic{page}}
   \noindent
   \begin{center}
   \framebox{
      \vbox{
    \hbox to 5.78in { {\bf \course} \hfill #2 }
       \vspace{4mm}
       \hbox to 5.78in { {\Large \hfill #5  \hfill} }
       \vspace{2mm}
       \hbox to 5.78in { {\it #3 \hfill #4} }
      }
   }
   \end{center}
   \vspace*{4mm}
}

\newcounter{pppp}
\newcommand{\prob}{\arabic{pppp}}  %problem number
\newcommand{\increase}{\addtocounter{pppp}{1}}  %problem number

%first argument desription, second number of points
\newcommand{\newproblem}[2]{
\ifnum\me=0
\ifnum\prob>0 \newpage \fi
\increase
\setcounter{page}{1}
\handout{\name, Homework \num, Problem \arabic{pppp}}{\today}{Name: \name}{Due:
\due}{Solutions to Problem \prob\ of Homework \num\ (#2)}
\else
\increase
\section*{Problem \num-\prob~(#1) \hfill {#2}}
\fi
}

%\newcommand{\newproblem}[2]{\increase
%\section*{Problem \num-\prob~(#1) \hfill {#2}}
%}

\def\squarebox#1{\hbox to #1{\hfill\vbox to #1{\vfill}}}
\def\qed{\hspace*{\fill}
        \vbox{\hrule\hbox{\vrule\squarebox{.667em}\vrule}\hrule}}
\newenvironment{solution}{\begin{trivlist}\item[]{\bf Solution:}}
                      {\qed \end{trivlist}}
\newenvironment{solsketch}{\begin{trivlist}\item[]{\bf Solution Sketch:}}
                      {\qed \end{trivlist}}
\newenvironment{code}{\begin{tabbing}
12345\=12345\=12345\=12345\=12345\=12345\=12345\=12345\= \kill }
{\end{tabbing}}

%\newcommand{\eqref}[1]{Equation~(\ref{eq:#1})}

\newcommand{\hint}[1]{({\bf Hint}: {#1})}
%Put more macros here, as needed.
\newcommand{\room}{\medskip\ni}
\newcommand{\brak}[1]{\langle #1 \rangle}
\newcommand{\bit}[1]{\{0,1\}^{#1}}
\newcommand{\zo}{\{0,1\}}
\newcommand{\C}{{\cal C}}

\newcommand{\nin}{\not\in}
\newcommand{\set}[1]{\{#1\}}
\renewcommand{\ni}{\noindent}
\renewcommand{\gets}{\leftarrow}
\renewcommand{\to}{\rightarrow}
\newcommand{\assign}{:=}

\newcommand{\AND}{\wedge}
\newcommand{\OR}{\vee}

\newcommand{\For}{\mbox{\bf For }}
\newcommand{\To}{\mbox{\bf to }}
\newcommand{\Do}{\mbox{\bf Do }}
\newcommand{\If}{\mbox{\bf If }}
\newcommand{\Then}{\mbox{\bf Then }}
\newcommand{\Else}{\mbox{\bf Else }}
\newcommand{\While}{\mbox{\bf While }}
\newcommand{\Repeat}{\mbox{\bf Repeat }}
\newcommand{\Until}{\mbox{\bf Until }}
\newcommand{\Return}{\mbox{\bf Return }}


\begin{document}

\ifnum\me=0
%\handout{PS\num}{\today}{Name: **** INSERT YOU NAME HERE ****}{Due:
%\due}{Solutions to Problem Set \num}
%
%I collaborated with *********** INSERT COLLABORATORS HERE (INDICATING
%SPECIFIC PROBLEMS) *************.
\fi
\ifnum\me=1
\handout{PS\num}{\today}{Name: Yevgeniy Dodis}{Due: \due}{Solution
{\em Sketches} to Problem Set \num}
\fi
\ifnum\me=2
\handout{PS\num}{\today}{Lecturer: Yevgeniy Dodis}{Due: \due}{Problem
Set \num}
\fi

\newproblem{Text Alignment}{6 points}

Using dynamic programming, find the optimum printing of the text {\em
``Not all those who wander are lost''}, i.e. $\ell_1=3, \ell_2 = 3,
\ell_3 =5, \ell_4 = 3, \ell_5 = 6, \ell_6 = 3, \ell_7=4$, with line
length $L=14$ and penalty function $P(x) = x^3$. Will the optimal
printing you get be consistent with the strategy ``print the word on
as long as it fits, and otherwise start a new line''? Once again, you
have to actually find the alignment, as opposed to only finding its
penalty.

\ifnum\me<2
\begin{solution}
We will first initialize array m[7] with values infinity.
Suppose $m_i$ holds the optimal penalty up to i words. Initially, we will try to minimize it for $m_1$.\\ 
1\\
We will fill the first one into first row
"Not \_ \_ \_ \_ \_ \_ \_ \_ \_ \_ \_"
Penalty will be $11^3$ = 1331\\
therefore array after first iteration will be 
[1331, $\infty, \infty, \infty, \infty, \infty, \infty$]\\
\\
Now for $m_2$\\
There are two options\\
1. Include "all" in second row 
then m[2] = m[1] + p(2) = 1331 + $(11)^3$ = 2662\\
2. Include "all" in first row, then the row will be "Not all". There are 7 spaces after "all"
Therefore penalty m[2] = 7*7*7 = 343
m[2] = min(2662,343)\\
Therefore array after second iteration will be 
[1331, 343, $\infty, \infty, \infty, \infty, \infty$]\\
and alignment is 
"Not All"\\
\\
Now for $m_3$, there are again two options\\
1. Include "those" in second row, alignment will be\\
"Not all \_ \_ \_ \_ \_ \_ \_"\\
"those \_ \_ \_ \_ \_ \_ \_ \_ "\\
then m[3] = m[2] + p = $343 + 7^3 + 8^3$ = 1198\\ 
2. Include "those" in the first row itself\\
"Not all those \_"\\
then m[3] = $1^3$ = 1\\
Therefore array after thirs iteration will be 
$[1331, 343, 1, \infty, \infty, \infty, \infty$]\\
and alignment is \\
"Not all those \_"\\
\\
Now for $m_4$, different cases will be\\
1. Include only "who" in new row\\
Alignment will be \\
"Not all those \_"\\
"who \_ \_ \_ \_ \_ \_ \_ \_ \_ \_ \_"\\
m[4] = m[3] + $11^3$\\
m[4] = 1 + 1331 = 1332\\
2. Include "those who" in new row\\
Alignment will be\\
"Not all \_ \_ \_ \_ \_ \_ \_"\\
"those who \_ \_ \_ \_ \_"\\
m[4] = m[2] + $5^3$ = 343 + 125 = 468\\
3. Include "all those who" in new row\\
Alignment will be\\
"Not \_ \_ \_ \_ \_ \_ \_ \_ \_ \_ \_"\\
"all those who \_"\\
m[4] = m[1] + $1^3$ = 1331 + 1 = 1332\\
\\
m[4] = min(1332,468,1332) = 468\\
optimal alignment is \\
"Not all \_ \_ \_ \_ \_ \_ \_"\\
"those who \_ \_ \_ \_ \_"\\
Array is 
$[1331, 343, 1, 468, \infty, \infty, \infty$]\\
\\
\\
Now for $m_5$, different cases will be\\
1. Include "wander" in new row\\
Alignment will be
"Not all \_ \_ \_ \_ \_ \_ \_"\\
"those who \_ \_ \_ \_ \_"\\
""wander \_ \_ \_ \_ \_ \_ \_ \_ \_ ""\\
m[5] = m[4] + $8^3$ = 468 + 512 = 980\\
\\
2. "Not all those \_"\\
"who wander \_ \_ \_ \_ "\\
m[5] = m[3] + $4^3$\\
m[5] = 1 +  64 = 65\\
\\
m[5] = min (980,65) = 65\\
Alignment will be
"Not all those \_"\\
"who wander  \_ \_ \_ \_"\\
Array will be \\
$[1331, 343, 1, 468, 65, \infty, \infty$]\\
\\
Now for $m_6$, different cases will be\\
1. Include "are" in new line
Alignment will be
"Not all those \_"\\
"who wander \_ \_ \_ \_ "\\
"are \_ \_ \_ \_ \_ \_ \_ \_ \_ \_ \_"\\
m[6] = m[5] + $11^3$ = 65 + 1331 = 1396\\
2. Include "wander are" in new line
Alignment will be\\
"Not all \_ \_ \_ \_ \_ \_ \_"\\
"those who \_ \_ \_ \_ \_"\\
"wander are  \_ \_ \_ \_"\\
m[6] = m[4] + $4^3$ = 468 + 64 = 532\\
3. Include "who wander are" in new line
Alignment will be\\
"Not all those \_ "
"who wander are "
m[6] = m[3] + 0 = 1+0 = 1\\
\\
m[6] = min(1457,593,2) = 1\\
And alignment is\\
"Not all those \_ "
"who wander are "
Array is $[1331, 343, 1, 468, 65, 1, \infty$]\\
\\
Now for $m_7$, different cases will be\\
1. Include "lost" in new row\\
Alignment will be\\
"Not all those \_ "
"who wander are "
"lost \_ \_ \_ \_ \_ \_ \_ \_ \_ \_"\\
m[7] = m[6] + $10^3$ = 1001
\\
2. Include "are lost" in new row\\
Alignment will be\\
"Not all those \_"\\
"who wander \_ \_ \_ \_ "\\
"are lost \_ \_ \_ \_ \_ \_"\\
m[7] = m[5] + $6^3$ = 65 + 216 = 281\\
m[7] = min(1001,281) = 281\\
Final array is $[1331, 343, 1, 468, 65, 1, 281$]\\
\\
\textbf{Optimal alignment is \\
"Not all those \_"\\
"who wander \_ \_ \_ \_ "\\
"are lost \_ \_ \_ \_ \_ \_"\\}



\end{solution}
\fi

\newproblem{Shortest Common Super-sequence}{20 points}

Let $X[1 \ldots m]$ and $Y[1 \ldots n]$ be two given arrays. A common supersequence of $X$ and $Y$ is an array $Z[1\ldots k]$ such that $X$ and $Y$ are both subsequences of $Z[1 \ldots k]$. Your goal is to find the {\em shortest} common super-sequence (SCS) $Z$ of $X$ and $Y$, solving the following sub-problems.

\begin{itemize}
 \item[(a)] (4 points) First, concentrate on finding only the length $k$ of $Z$. Proceeding similarly to the longest common subsequence problem, define the appropriate array $M[0\ldots m,0\ldots n]$ (in English), and the write the key recurrence equation to recursively compute the values $M[i,j]$ depending on some relation between $X[i]$ and $Y[j]$. Do not forget to explicitly write the base cases $M[0,j]$ and $M[i,0]$, where $1\le i\le m, 1\le j\le n$.

\ifnum\me<2
\begin{solution}
M will be a 2 dimensional array of size (m+1)X(n+1) where each entry in $M_{i,j}$ gives the length of shortest common supersequence of $X_1X_2\ldots X_i$ and $Y_1Y_2 \ldots Y_j$\\. i will vary from 1 to m and j will vary from 1 to n\\
M[0,j] gives the length of shortest common supersequence of $Y_1Y_2 \ldots Y_j$ which is equal to j itself\\
M[i,0] gives the length of shortest common supersequence of $X_1Y_2 \ldots X_j$ which is equal to i itself\\
Therefore for base case\\
M[0,j] = j \hspace{10mm} for $1 \leq j \leq n$\\
M[i,0] = i \hspace{10mm} for $1 \leq i \leq m$\\
and M[0][0] = 0 since length of common supersequence of two null words is 0\\
Now, \textbf{Recursive formula will be\\
M[i,j] =  1 + M[i-1][j-1]	\hspace{30mm} if $X_i=Y_j$\\
M[i,j] =  1 + min (M[i-1,j] , M[i][j-1])	\hspace{7mm} if $X_i$ is not equal to $Y_j$\\ }
\\
If $X_i = Y_j$ then two characters are equal and the character should be added to longest common subsequence\\
Otherwise, if two characters are not equal, then both the characters should be added but there will be two cases\\
(a). $X_i$ before $Y_j$ = total length will be 1 (for $Y_j$) + M[i,j-1]\\
(b). $Y_j$ before $X_i$ = total length will be 1 (for $X_i$) + M[i-1,j]\\
We will take the minimum of these two cases for the optimal solution\\

\end{solution}
\fi

\item[(b)] (5 points) Translate this recurrence equation into an explicit bottom-up
$O(mn)$ time algorithm that computes the length of the shortest common supersequence of $X$ and $Y$.

\ifnum\me<2
\begin{solution}
\\
\\
\{\\
Let M be new array of size (m+1)X(n+1) all initialized to all zeros\\
Setting all the base cases\\
M[0][0] = 0\\
for i = 1 to m\\
M[i][0] = i\\
for j = 1 to n\\
M[0][n] = j\\
\\
for(i = 1 to m)\\
\{\\
for(j = 1 to n)\\
\{\\
if($X_i = Y_j$)\\
M[i][j] = 1 + M[i-1][j-1]\\
else\\
M[i][j] = 1 + MINIMUM ( M[i,j-1] , M[i-1, j]) \\
\\
\} \%End First For Loop \\
\} \%End Second For loop\\
\} \%End whole function\\
\end{solution}
\fi

\item[(c)] (5 points) Find the SCS of  $X=BARRACUDA$ and $Y=ABRACADABRA$.
(Notice, you need to find the actual SCS, not only its length.)

\ifnum\me<2
\begin{solution}
In this case, along with the values of length, we will also store a direction character which will tell information about the previous character from where we have arrive to current position. For example, if we are filling the values of m[i][j], then three possible cells from where we would have arrived at m[i][j] are m[i-1][j-1], m[i-1][j] and m[i[j-1].
Possible characters are :\\
d: Diagnol  (M[i-1][j-1] \hspace{5mm} l: Left  (M[i][j-1]) \hspace{5mm} u: Up  (M[i-1][j])\\
So the conditions now will be\\
if($X_i = Y_j$)\\
M[i][j] = 1 + M[i-1][j-1] \& S[i][j] = d\\
else if($M[i][j-1] \leq M[i-1][j]$)\\
M[i][j] = 1 + M[i][j-1] \& S[i][j] = l\\
else\\
M[i][j] = 1 + M[i-1][j] \& S[i][j] = u\\
For base cases, S[i][j] is initialized as follows\\
For i=1 to m, S[i,0] = u ;\\
For j=1 to n, S[0,j] = l ;\\

The calculated table for two given words will be of size (9+1)X(11+1)\\
To show S[i][j] in more understandable form, I have concatenated it to the M[i][j] value\\
\\
Initially length=M[m][n], i=m, j=n, scs = char[legth]\\
while($i\geq0$ \& $j\geq0$)\\
ch = S[i][j];\\
\\
if(ch=='d')\\
scs[length - -] = X[i];\\
i=i-1 \& j=j-1\\
\\
if(ch=='l')\\
scs[length - -] = Y[j];\\
j=j-1;\\
\\
else
scs[length - -] = X[i];\\
i=i-1;\\
\\
end while\\
\\
for i=1:length\\
print scs[i]\\
end for\\

This will print final word\\
SCS = \textbf{B A R B R A C U A D A B R A}\\

\end{solution}
\fi

\item[(d)] (6 points) Show that the length $k$ of the array $Z$ computed in
part (a) satisfies the equation $k = m + n - \ell$, where $\ell$ is the length of the longest common {\em subsequence} of $X$ and $Y$. \\
\hint{Use the recurrence equation in part (a), then combine it with a similar recurrence equation for the LCS, and then use induction. There the following identity is very handy: $\min(a,b)+\max(a,b) = a+b$.}
\end{itemize}

\ifnum\me<2
\begin{solution}
\vspace{100mm}\\
\end{solution}
\fi

\newproblem{Babysitting Dillemma}{18 points}

You are a CFO of a baby sitting company, and got a request to baby sit
$n$ children one day. You can hire several babysitters for a day for a
fixed cost $B$ per babysitter. Also, you can assign an arbitrary
number of children $i\ge 1$ to a babysitter. However, each parent will
only pay some amount $p[i]$ if his child is taken care of by a
babysitter who looks after $i$ children. For example, if $n=7$ and you
hire $2$ babysitters who looks after $3$ and $4$ children,
respectively, you revenue is $3p[3] + 4p[4] - 2B$.

Given $B,n,p[1],\ldots, p[n]$, your job is to assign children to
babysitters as to maximize your total profit. Namely, you want to find
an optimal number $k$ and an optimal partition $n=n_1+\ldots+n_k$ so
as to maximize revenue $R = n_1\cdot p[n_1]+\ldots+n_k \cdot p[n_k] -
k\cdot B$.

\begin{itemize}
\item[(a)] (5 points) Let $R[i]$ denote the optimum revenue you can get by
looking after $i$ children. E.g., $R[0]=0$, $R[1] = p[1]-B$, $R[2] =
\max(2p[1]-2B, 2p[2]-B)$, etc. Write a top-down recursive formula for
$R[n]$ in terms of values $R[j]$ for $j<n$.

\ifnum\me<2
\begin{solution}\\
R[n] = maximum( R[n-1] + R[1] , R[n-2] + R[2],  $\ldots$ , R[1] + R[n-1], n*P[n] - B)\\
\\
R[n] = maximum (R[n-q] + R[q], n*p[n] - B) Where q varies from 1 to n/2\\

\end{solution}
\fi

\item[(b)] (5 points) Write a top-down recursive procedure with memorization which
will compute $R[n]$. Analyze the running time of your procedure in the
$\Theta(\cdot)$ notation.

\ifnum\me<2
\begin{solution}
OptimalBSTopDown\\
\{\\
Initialize an array r[n] with all values = -$\infty$\\
OptimalBSTopDown-AUX(p,n,r)\\
\}\\
\\
OptimalBSTopDown-AUX(p,n,r)\\
\{\\
if($r[n] \geq 0$)\\
return r[n]\\
else\\
if (n==0)
 temp=0;\\
 else \{ \\
 $temp=n*p[n] - B$\\
 \\
 for q=1 to n/2\\
 temp = max(temp, OptimalBSTopDown-AUX(p,n-q,r) + OptimalBSTopDown-AUX(p,q,r)) \\
end for\\
r[n]=temp\\
return temp\\ 
\}\\
\\
Analysis of running time\\
Initially when OptimalBSTopDown-AUX is called, it is called with n\\
To calculate r[n], the inner loop will be run at least n/2 times\\
\\
Now different branches will be \\
T(1) + T(n-1), T(2) + T(n-2), T(3) + T(n-3), $\ldots$, T(n/2) + T(n/2)\\
But we know that T(1) will be run only once so is T(2) and all other. As when it is again called for T(1), it will just return from the function form the first step and no work will be done\\
So you can see that for size n, loop runs n/2 times\\
for size n-1, loop runs (n-1)/2 times\\
Similarly for size 1, loop runs 0 times\\
Hence total work done or total complexity will be \\
0 + 1 + $\ldots$ + (n-1)/2 + n/2 = $\Theta (n^2)$

Hence complexity of algorithm will be $\Theta (n^2)$\\
\end{solution}
\fi


\item[(c)] (5 points) Write an iterative bottom-up variant of the same
procedure.

\ifnum\me<2
\begin{solution}
Bootom-Up-Baby-Sitter(p,n)\\
\{\\
let r[n] be a new array \\
r[0]=0\\
for j = 1 to n\\
temp = n*p[n] - B\\
for q = 1 to j/2\\
temp = max ( r[j-q] + r[q],temp )\\
end for\\
r[j] = temp\\
end for\\
return r[n]\\
\}\\
\\
Time complexity of this solution will be $\Theta (n^2)$ as \\
For j=1, inner loop will run 0 times\\
For j=2, inner loop will run 1 time\\
.\\
.\\
For j=n, inner loop will run n/2 times\\
Total T(n) = 0 + 1 + 2 + 3 + $\ldots$ + n/2 = $(n^2 + n)/4$ = $  \Theta (n^2) $ \\
\\

\end{solution}
\fi

\item[(d)] (3 points) Explain how to augment your procedure (either in part (b) or
(c)) to also compute the optimal number of babysitters $k$ and the
actual partition of children. Either English or pseudocode will work.

\ifnum\me<2
\begin{solution}
For storing the partition info, we will have to create a new array say s[n]\\
We will make some changes in the code\\
1. When we initialized temp, we have to add one line in the initializisation it as\\
   temp = n*p[n] - B
   s[n] = n\\
\\
2.    Now inside the second nested for loop in which q varies from 1 to n/2\\
We will re write the lines as\\
if (temp < r[n-q] + r[q])\\
temp = r[n-q] + r[q]\\
s[n] = q\\
\\
Now for retrieving back the actual partition of children we will add the below-mentioned function and call it as print-optimalset(s,n)\\
print-optimalset(s,n)\\
\{\\
if(n$\leq$0)\\
print("Done printing")\\
else if (s[n] = n)\\
print(s[n]\\
else\\
print-optimalset(s,n-s[n])\\
print-optimalset(s,s[n])\\
\}\\

\end{solution}
\fi

\end{itemize}

\newproblem{Dividing Chocolate}{10 points}

You have $m\times n$ chocolate bar. You are also given a matrix
$\{p[i,j]\mid 1\le i\le m, 1\le j\le n\}$ telling you the price of the
$i\times j$ chocolate bar. You are allowed to repeat the following
procedure any number of times, starting initially with the single big
$m\times n$ piece you have. Take one of the pieces you have and split
it into two pieces by cutting it either vertically or
horizontally. Say, $m=5$, $n=4$. You may first choose to split it into
two pieces of size $3\times 4$ and $2\times 4$. Then you may take the
$3\times 4$ piece and split it into two pieces $3\times 2$ and
$1\times 2$. Finally, you may take the previous $2\times 4$ piece and
split it into two $1\times 4$ pieces. If you stop, you have
four pieces of sizes $3\times 2$, $1\times 2$, $1\times 4$ and
$1\times 4$, which you can sell for $p[3,2]+p[1,2]+2p[1,4]$.
You goal is to find a partition maximizing your total profit.

\begin{itemize}

\item[(a)] (5 points) Let $C[i,j]$ be the largest profit you can get by splitting
an $i\times j$ piece, where $0\le i\le m$, $0\le j\le n$ and we set
$C[i,0] = C[0,j]=0$. Write a recursive formula for $C[m,n]$ in terms
of values $C[i,j]$, where either $i<m$ or $j<n$.

\ifnum\me<2
\begin{solution}\\
C[m][n] = MAX( MAX(C[i,n] + C[m-i,n]) where $1\leq i \leq m$ , MAX (C[m,j] + C[m,n-j]) where $1\leq j \leq n$ , P[m,n])\\
In short\\
C[m,n] = MAX (C[i,n] + C[m-i,n] , C[m,j] + C[m,n-j], P[m][n]) where\\
$1\leq i < m$ and $1 \leq j < n $\\
To remove redundancy we can optimize this formula by changing the limits of i and j\\
$1 \leq i \leq m/2$ and $ 1 \leq j \leq n/2$\\
\end{solution}
\fi

\item[(b)] (5 points) Write a bottom-up procedure to compute $C[m,n]$ and
analyze its running time as a function of $m$ and $n$.

\ifnum\me<2
\begin{solution}
Bottom-up-chocolate(p,m,n)\\
\{\\
let c[m,n] be new array\\
\% base cases\\
for i = 1 to m\\
c[i][0] = 0\\
end for\\
for j = 1 to n\\
c[0][j] = 0\\
end for\\
\\
\\
for i = 1 to m\\
for j = 1 to n\\
temp = p[i][j];\\
for k = 1 to m/2\\
temp = max ( temp, c[k,n] + c[m-k,n])\\
end for\\
for k = 1 to n/2\\
temp = max (temp, c[m,k] + c[m,n-k])\\
end for\\
c[i][j] = temp\\
end for\\
end for\\
return c[m,n]\\
\}\\
\\
To analyze the Time Complexity\\
Outermost loop runs m times\\
Second to Outermost loop runs n times\\
third to innermost loop runs m/2 + n/2 times\\
Therefore total complexity will be m*n*(m/2+n/2) = mn(m+n) = $O(mn(m+n))$\\

\end{solution}
\fi

\end{itemize}



\end{document}
