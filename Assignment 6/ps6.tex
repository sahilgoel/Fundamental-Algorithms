%%me=0 student solutions, me=1 - my solutions, me=2 - assignment
\def\me{0}
\def\num{6}  %homework number
\def\due{Wednesday, October 15}  %due date
\def\course{CSCI-GA.1170-001/002 Fundamental Algorithms} %course name
\def\name{Sahil Goel}
%
\iffalse
INSTRUCTIONS: replace # by the homework number.
(if this is not ps#.tex, use the right file name)

  Clip out the ********* INSERT HERE ********* bits below and insert
appropriate TeX code.  Once you are done with your file, run

  ``latex ps#.tex''

from a UNIX prompt.  If your LaTeX code is clean, the latex will exit
back to a prompt.  To see intermediate results, type

  ``xdvi ps#.dvi'' (from UNIX prompt)
  ``yap ps#.dvi'' (if using MikTex in Windows)

after compilation. Once you are done, run

  ``dvips ps#.dvi''

which should print your file to the nearest printer.  There will be
residual files called ps#.log, ps#.aux, and ps#.dvi.  All these can be
deleted, but do not delete ps1.tex. To generate postscript file ps#.ps,
run

  ``dvips -o ps#.ps ps#.dvi''

I assume you know how to print .ps files (``lpr -Pprinter ps#.ps'')
\fi
%
\documentclass[11pt]{article}
\usepackage{amsfonts,amsmath}
\usepackage{latexsym}
\setlength{\oddsidemargin}{.0in}
\setlength{\evensidemargin}{.0in}
\setlength{\textwidth}{6.5in}
\setlength{\topmargin}{-0.4in}
\setlength{\textheight}{8.5in}

\newcommand{\handout}[5]{
   \renewcommand{\thepage}{#1, Page \arabic{page}}
   \noindent
   \begin{center}
   \framebox{
      \vbox{
    \hbox to 5.78in { {\bf \course} \hfill #2 }
       \vspace{4mm}
       \hbox to 5.78in { {\Large \hfill #5  \hfill} }
       \vspace{2mm}
       \hbox to 5.78in { {\it #3 \hfill #4} }
      }
   }
   \end{center}
   \vspace*{4mm}
}

\newcounter{pppp}
\newcommand{\prob}{\arabic{pppp}}  %problem number
\newcommand{\increase}{\addtocounter{pppp}{1}}  %problem number

%first argument desription, second number of points
\newcommand{\newproblem}[2]{
\ifnum\me=0
\ifnum\prob>0 \newpage \fi
\increase
\setcounter{page}{1}
\handout{\name, Homework \num, Problem \arabic{pppp}}{\today}{Name: \name}{Due:
\due}{Solutions to Problem \prob\ of Homework \num\ (#2)}
\else
\increase
\section*{Problem \num-\prob~(#1) \hfill {#2}}
\fi
}

%\newcommand{\newproblem}[2]{\increase
%\section*{Problem \num-\prob~(#1) \hfill {#2}}
%}

\def\squarebox#1{\hbox to #1{\hfill\vbox to #1{\vfill}}}
\def\qed{\hspace*{\fill}
        \vbox{\hrule\hbox{\vrule\squarebox{.667em}\vrule}\hrule}}
\newenvironment{solution}{\begin{trivlist}\item[]{\bf Solution:}}
                      {\qed \end{trivlist}}
\newenvironment{solsketch}{\begin{trivlist}\item[]{\bf Solution Sketch:}}
                      {\qed \end{trivlist}}
\newenvironment{code}{\begin{tabbing}
12345\=12345\=12345\=12345\=12345\=12345\=12345\=12345\= \kill }
{\end{tabbing}}

%\newcommand{\eqref}[1]{Equation~(\ref{eq:#1})}

\newcommand{\hint}[1]{({\bf Hint}: {#1})}
%Put more macros here, as needed.
\newcommand{\room}{\medskip\ni}
\newcommand{\brak}[1]{\langle #1 \rangle}
\newcommand{\bit}[1]{\{0,1\}^{#1}}
\newcommand{\zo}{\{0,1\}}
\newcommand{\C}{{\cal C}}

\newcommand{\nin}{\not\in}
\newcommand{\set}[1]{\{#1\}}
\renewcommand{\ni}{\noindent}
\renewcommand{\gets}{\leftarrow}
\renewcommand{\to}{\rightarrow}
\newcommand{\assign}{:=}

\newcommand{\AND}{\wedge}
\newcommand{\OR}{\vee}

\newcommand{\For}{\mbox{\bf For }}
\newcommand{\To}{\mbox{\bf to }}
\newcommand{\Do}{\mbox{\bf Do }}
\newcommand{\If}{\mbox{\bf If }}
\newcommand{\Then}{\mbox{\bf Then }}
\newcommand{\Else}{\mbox{\bf Else }}
\newcommand{\While}{\mbox{\bf While }}
\newcommand{\Repeat}{\mbox{\bf Repeat }}
\newcommand{\Until}{\mbox{\bf Until }}
\newcommand{\Return}{\mbox{\bf Return }}


\begin{document}

\ifnum\me=0
\fi
\ifnum\me=1
\handout{PS\num}{\today}{Name: Yevgeniy Dodis}{Due: \due}{Solution
{\em Sketches} to Problem Set \num}
\fi
\ifnum\me=2
\handout{PS\num}{\today}{Lecturer: Yevgeniy Dodis}{Due: \due}{Problem
Set \num}
\fi


\newproblem{Find the Josephus Permutation} {10 points}

According to Josephus' account of the siege of Yodfat, he and his $n$
comrade soldiers were trapped in a cave, the exit of which was blocked
by Romans.  They chose suicide over capture and decided that they
would form a circle and start killing themselves using ``step'' of
size $m$; i.e., starting the count with some fixed person (called
``number $1$''), every $m$-th person is killed, after which the
suicides continue with the remaining people on the (now smaller)
circle.  For example for $n = 8$ and $m = 3$, the order in which the
people are killed is $(3, 6, 1, 5, 2, 8, 4, 7)$. We want an $O(n \log
n)$ algorithm to find the order in which the soldiers were killed.
\hint{Use {\em augmented} 2-3 trees.}

\ifnum\me<2
\begin{solution}
This problem can be solved using Augmented 2-3 trees by storing the v.num for every node which stores number of nodes in v's subtree.\\
\\
Since we know, if we store v.num with every node, we can find out any kth rank element in time $O(\log n)$. We will use this property in our algorithm, to find any kth rank element.\\  
\\
We will also use a ptr root which will always point at root and it can check whether the tree is empty or not and also number of elements present in the tree at any given time.\\
\\
Overall Idea: Initialize rank=k(First element to be deleted or first person to die). After the first element is deleted, the next element that should be deleted should have rank = previous element rank + k, but when the previous element is deleted, the new rank will be =previous element rank + k - 1(because of deletion of that node). Now we need to check if this rank is lower than the total number of nodes in tree or not. If it is lower then continue, else take rank =  mod(rank, root.num) and delete the element with rank=rank. Continue until root.num is not equal to zero or tree is not empty.\\
\\
Assume create2-3Augmentedtree is a procedure which creates an 2-3 Augmented tree for elements of Arr[ ] which contains all the numbers from 1 to n. This procedure also stores v.num value for every node. This procedure returns the root of the tree\\
\\
Also assuming that whenever a node is deleted, all the v.num values are updated.
\\
suicideorder(Arr[],k)\\
\{\\
root = create2-3Augmentedtree(Arr[]);\\
"rank"=k;\\
while(root is not equal to NULL)\\
\hspace{55mm}  Print element having rank "rank"\\
\hspace{55mm}	 Delete element having rank "rank"\\
\hspace{55mm}	 "rank" = mod("rank" + $k - 1$, root.num)\\ 
\hspace{55mm} if(rank==0) rank=root.num\\
end while\\
\}\\
\\
\\
This will print the order in which the persons will die\\
\\
Example\\
Suppose n=8 and k=3\\
Initially k=3, elements in tree are {1,2,3,4,5,6,7,8}\\
print element with rank 3 : =	\{3\}\\
now element with rank 3 will be deleted\\
\\
Now, root.num=7 and tree = \{1,2,4,5,6,7,8\}\\
new rank=mod(rank + k-1,7)\\
rank = mod(3 + 3 - 1 , 7)\\ 
rank = 5\\
print element with rank 5 := \{6\}\\
Now element with rank 5 will be deleted\\
\\
Now, root.num=6 and tree = \{1,2,4,5,7,8\}\\
new rank = mod(5 + 3 - 1,6) = 1\\
rank=1\\
print element with rank 1 := \{1\}\\
Now element with rank 1 will be deleted\\
\\ 
Now, root.num=5 and tree = \{2,4,5,7,8\}\\
new rank = mod(1 + 3 - 1, 5) = 3\\
print element with rank 3 := \{5\}\\
Now element with rank 3 will be deleted\\
\\
Now, root.num=4 and tree = \{2,4,7,8\}\\
new rank = mod(3 + 3 - 1, 4) = 1\\
print element with rank 1 := \{2\}\\
Now element with rank 1 will be deleted\\
\\
Now, root.num=3 and tree = \{4,7,8\}\\
new rank = mod(1 + 3 - 1, 3) = 0\\
rank = root.num = 3\\
print element with rank 3 := \{8\}\\
Now element with rank 3 will be deleted\\
\\
Now, root.num=2 and tree = \{4,7\}\\
new rank = mod(3 + 3 - 1, 2) = 1\\
print element with rank 1 := \{4\}\\
Now element with rank 1 will be deleted\\
\\
Now, root.num=1 and tree = \{7\}\\
new rank = mod(1+3-1,1) = 0\\
rank = root.num = 1\\
print element with rank 1 := \{7\}\\
Now element with rank 7 will be deleted\\
\\
\\
\\
\textbf{As you can observe the order of deletion is "3, 6 ,1, 5 , 2, 8, 4, 7" which is the correct order}\\
\\
Lets calculate the complexity of solution\\
1. create2-3Augmentedtree will take time $O(n \log n)$ as to insert one element it takes time $O(\log n)$ and hence to insert n elements it will be $O(n \log n)$\\
2. deleting a single node takes time $O(\log n)$, and since this delete is in while loop which will be called for n times, total time in the loop is $O(n \log n)$\\
\\
Total time  T = T(step 1) + T(step 2) = $O(n \log n)$ +  $O(n \log n)$\\
T = $O(n \log n)$\\


\end{solution}
\fi


\newproblem{Computing Less}{18 Points}

Assume you are given a binary search tree $T$ on $n$ elements, whose
height is $h$. As usual, let $v.key$ denote the key value of node $v$,
and recall than the left sub-tree of every node $v$ only contains
elements less than {\em or equal to} $v.key$. (In particular, our tree
might contain several elements having the same key.)

\begin{itemize}

\item[(a)] (5 points)
Using a slight modification of the {\sc PostOrder-Tree-Walk}
procedure, argue that in time $\Theta(n)$ you can compute, for every
node $v$, the number of nodes (call it $v.less$) {\em in $v$'s
sub-tree} which are {\em strictly} less than $v.key$.
Write a pseudocode for the resulting procedure {\sc Fill-Less}$(T)$,
which will fill in the values $v.less$ for all the nodes of $T$
in time $O(n)$.
\\
\hint{Remember than some elements in the left sub-tree of $v$ might
be equal to $v.key$, and should not be included in the count for
$v.less$. Also do not forget to cover the base case.}

\ifnum\me<2
\begin{solution}\\
Assuming T points to the root of the Tree initially. This problem can be solved by augmenting BST tree nodes with these values\\
(a). v.num : It stores number of nodes in its subtree\\
(b). v.max and count : It stored the maximum value in its subtree and its count\\
Now, these  values can be calculated in O(n) time\\
\\
To calculate v.num in time O(n)\\
\textbf{PTWnum(T)}\\
\{\\
if(T==NULL)\\
return -1;\hspace{10mm} \% -1 because we counted this node so we will have to subtract\\
PTWnum(T.left);\\
PTWnum(T.right);\\
T.num = T.left.num + T.right.num + 2;\hspace{2mm} \%2 to add left and right child\\
return T.num;
\}\\
\\
To calculate v.maxc in time O(n). There can be 3 cases for each node\\
(a). If node v has a right subtree, then the maximum value for subtree under this node will lie in its right subtree. The maximum value then can be its right child or T.right.max, which is maximum of right child's subtree. If the maximum is equal to the right child, then count=1 as it will have occurred first time else the count will be what it was before i.e. T.right.count  \\
\\
(b). If node v doesn't have a right subtree and Its left child is equal to the maximum of left child's subtree, then we need to assign T.max as its left child and T.count = T.left.count+1\\
\\
(c). If node v doesn't have a right subtree and its left child is smaller or greater than the maximum
then we will assign T.max=T.left.max and T.count=T.left.count\\
\\
Initially max=-infinity and count=0\\
\textbf{PTWmaxnc(T)}\\
\{\\
if(T==NULL)\\
return\\
PTWmaxnc(T.left);\\
PTWmaxnc(T.right);\\
if(T.right!=NULL)\hspace{10mm} \% Case (a)\\
T.max=maximum(T.right.max,T.right);\\
\hspace{5mm} if(T.right.max>=T.right) then \\T.count = T.right.count\\
\hspace{5mm} else T.count=1\\
else if(max=T.left) \hspace{10mm} \% Case (b) \\
T.count=T.left.count+1\\
T.max=T.left\\
else \hspace{10mm} \% Case (c)\\
T.max=T.left.max;\\
T.count=T.left.count;\\
\}\\
\\
Now We have calculated all these values and now we will use these to calculate the less values for every node. Notice, for every node\\
there will be 2 cases now\\
if v.key = v.left.max 
v.less = v.left.less - v.left.count\\
else v.less = v.left.less\\
Also we have to add "1" if v.left.ket is not equal to v.key\\
v.less = v.less + 1\\
  


\textbf{Fill-less(T)}\\
\{\\
\% Base case :- When T is a leaf node  or when tree is empty or when only right subtree exists\%
if($T == NULL$)\\
return\\
\\
if ($T.left == NULL$ and $T.right==NULL$)\hspace{5mm} Leaf node\\
	T$.$less = 0\\
	return\\

if ($T.left == NULL$)\hspace{10mm} Only right subtree exists\\
	T$.$less = 0\\
	Fill-less($T.right$)\\
	return\\

else\\
Fill-less($T.left$)\\
Fill-less($T.right$)\\
\\
if($T.key = T.left.max$ )\\
T$.$less = T$.left.less - T.left.count$\hspace{5mm} Left subtree has "T.count" equal elements to T.key\\
else\\
T$.$less = T$.left.less$ \\
\\
\\
\% We still have to add 1 for left child if it not equal to T.key\\
if(T.key is not equal to T.left.key)
T.less=T.less+1;
\\
return\\
\}\\


\end{solution}
\fi

\item[(b)] (5 points) Now that each node $v$ contains the value $v.less$,
show that you keep maintaining this value for each successive {\sc
Insert} operation. Namely, write the pseudocode for the {\sc
Insert}$(T,value)$ procedure, which will insert a new (leaf) node $w$ into
$T$ with $w.key=value$. The running time of your procedure should be
$O(h)$, and it should correctly maintain all the $v.less$ values.\\
\hint{When going down the tree from a node $v$, when does the
insertion of $w$ increases the value $v.less$?}

\ifnum\me<2
\begin{solution}\\
Notice to insert an element in a BST, we initially search for element and find the exact position where it should be inserted. If value is less than or equal to any node v.key, then we go to the left subtree and if its greater than node v.key, than we go to right subtree. In this case there will be a slight modification. Whenever value is less than node v.key, we will increment v.less by 1 and go to left, whereas if it is equal to node v, we will still go left but not increment v.less. There is no change if we go to right\\


Insert(T,value)\\
\{\\
if(T==NULL)	\hspace{10mm} Found correct place to insert\\
T=new v;\\
T.key=value;\\
T.less=0;\\

if(value $<$ T.key)\\
T.less=T.less+1;\\
Insert(T.left, value);\\
\\
if(value $=$ T.key)\\
Insert(T.left,value);\\
\\
if(value $>$ T.key)\\
Insert(T.right, value)\\
\\
\}\\


\end{solution}
\fi

\item[(c)] (5 points) Assume now that we successfully maintain the $v.less$
field for both the {\sc Insert} and the {\sc Delete} Operation, and
also that all the key values are distinct. Show how to implement an
$O(h)$-time procedure {\sc MyRank}$(v)$, which will return the rank of
a node $v$ in the BST (i.e., if $v.key$ is the minimum of all the
values it will return $1$, or if it is the median it will return
$n/2$). Why can't we simply return $v.less+1$ in time $O(1)$?

\ifnum\me<2
\begin{solution}\\
We know, that if we go to left child of node all the nodes are less than the node and if we go to right child of node all the nodes are greater than node. Now in this case we will have to consider the reverse of traversing. Suppose A is parent of B. \\
1. If B is left child of A, then $A.key>B.key$, therefore rank of B will not be affected by A's right subtree because all the elements in A's right subtree will be greater than B.key\\
\\
2. If B is right child of A, then $A.key< B.key$ and we also know that all the elements in A's left subtree are also less than A.key, therefore all those elements will be less than B.key also. Hence rank = rank + 1 + number of elements less than A.key(A.less). We had to add one to consider A.key itself.\\

MyRank(v)\\
\{\\
temp = new v;\\
\% Initialize rank with 1 and temp with v.parent\\
rank = 1;
temp=v.parent;\\
while(temp is not NULL)\\
\{\\
if(v is left child of temp) \hspace{5mm} \%No computation required just traverse up\\
\{\\
v=temp;\\
temp=v.parent;\\
\}\\
\\
if(v is right child of temp) \hspace{5mm} \%It is greater than all elements in left subtree of parent\\
\{\\
rank = rank + temp.less + 1; \hspace{5mm} \%+1 because of parent itself\\
v=temp;\\
temp=v.parent;\\
\}\\
return rank;\\
\}\\

Since we are traversing up all the time and at each level we are performing work O(1), therefore total time complexity is O(1*h)= O(h)\\
\\
\\
We can't simply return v.less + 1, because it only gives the number of elements which are less than the key in its subtree, it doesn't consider smaller elements in other subtrees which might be present. Example:- if we do this, all the leaf nodes will have rank 1 which is clearly wrong.\\ 



\end{solution}
\fi

\item[(d)] (3 points) Instead of running the procedure {\sc MyRank} as
in part (c), is it possible to simply maintain --- in time $O(h)$ ---
the field $v.rank$ which will correctly compute the rank of $v$ after
each insertion and deletion? If yes, show how, if not, explain why
not.

\ifnum\me<2
\begin{solution}
No, it is not possible to maintain the field v.rank in time O(h) because, deletion of one element might change the rank of all the elements in the tree. For example, suppose if the minimum of tree is deleted, then all the other nodes rank should be decremented by 1 which will take time O(n). Also if suppose an element is added which is minimum of all, then every node rank should be incremented by 1 which is also O(n). To conclude, whenever we delete or insert a node, ranks of all the nodes might get affected(not just the in its subtree or its parents)., hence it is not possible to maintain rank in time O(h).\\
\end{solution}
\fi

\end{itemize}

\newproblem{Order Matters!}{14 (+2) points}

Assume we insert sequences of English letters into an empty 2-3-tree,
following the standard alphabetic ordering when making comparisons:
$A<B<C<\ldots < Z$.

\begin{itemize}
 \item[(a)] (3 points) Assume you insert the letters $R, E, L, A, T, I, O,N, S$
into a fresh 2-3-tree $T$ in that exact order. What is the
resulting height of $T$?

\ifnum\me<2
\begin{solution}
***************** INSERT PROBLEM \prob a SOLUTION HERE ***************
\end{solution}
\fi

\item[(b)] (4 points) Let $h$ be your answer to part (a). Is it
  possible to get less than $h$? If not, prove it. If yes, give the
  best ordering you can. What is the depth $h_{\min}$ you get?

\ifnum\me<2
\begin{solution}
***************** INSERT PROBLEM \prob b SOLUTION HERE ***************
\end{solution}
\fi

\item[(c)] (3 points) Can you achieve $h_{\min}$ with lexicographic
  order $A,E,I,L,N,O,R,S,T$?

\ifnum\me<2
\begin{solution}
***************** INSERT PROBLEM \prob c SOLUTION HERE ***************
\end{solution}
\fi

\item[(d)] (2 points) {\bf Extra credit:} Can you achieve $h_{\min}$
  for part (b) using an English word that is a permutation of $R, E,
  L, A, T, I, O,N, S$?

\ifnum\me<2
\begin{solution}
***************** INSERT PROBLEM \prob d SOLUTION HERE ***************
\end{solution}
\fi

\item[(e)] (4 points) Give an example of a 2-3-tree $T$ with exactly 5 leaves
and two distinct values $a,b$, such that one gets different final
trees if one first inserts $a$ into $T$ and then deletes $b$, instead
of first deleting $b$ from $T$ and then inserting $a$. Explain what
happened.

\ifnum\me<2
\begin{solution}
***************** INSERT PROBLEM \prob e SOLUTION HERE ***************
\end{solution}
\fi

\end{itemize}

\newproblem{Successor of an Element} {14 points}

Assume that you are given a 2-3 tree $T$ containing $n$ distinct elements.

\begin{itemize}
\item[(a)] (4 points) Show how to find the successor of a given
  element $x \in T$ in time $O(\log n)$.

\ifnum\me<2
\begin{solution}\\
Initially to search for element x and find a pointer to it\\
search(T,x)\\
\{\\
while(T.key not equal to x)\\
\hspace{10mm} if(T.key less than x)\\
\hspace{10mm} return element doesn't exist\\
\hspace{10mm} if(T.child is leaf node and T.child[1] $<>$ x and T.child[2] $<>$ x and T.child[3] $<>$x)
return -1;\\
\hspace{10mm} for i in 1..3 loop
\hspace{10mm} if(T.child[i] = x and T.child[i] is leaf node)\\
\hspace{10mm} return T.child[i];\\
\hspace{10mm} end for;\\
\hspace{10mm} for i in 1..3 loop\\
\hspace{10mm} if(T.child[i].key $\geq$ x)\\
\hspace{10mm}	 break;\\
\hspace{10mm} search(T.child[i],x)\\

\}\\
\\
Successor(T,x)
\{

pos = search (T,x);\\
if(pos has a right sibling)\\
return right sibling\\
else\\
\{\\
pos = pos.parent;\\
while(pos is rightmost child of its parent and pos != root)\\
pos=pos.parent;\\
end while\\
if(pos==root)\\
return "No successor element"\\
else\\
posright= right sibling of post;\\
while(posright is not equal to leaf node)\\
posright=postright.left;\\
end while\\
return posright\\
\}
\end{solution}
\fi

\item[(b)] (4 points) Show that if the input element $x$ is chosen
  {\em uniformly at random} from $T$, then your procedure from part
  (a) runs in {\em expected} time  $O(1)$.

\ifnum\me<2
\begin{solution}\\
Observe, 
There will be one element for which we will have to traverse the whole tree till root and then down to leaf(Rightmost element of leftsubtree to Leftmost of element of right subtree).\\
There will be two Elements for which we will have to traverse till height h-1 and then down to leaf.\\
Similarly\\
1	\hspace{10mm} 	2*(h-0)\\
2	\hspace{10mm}		2*(h-1)\\
4	\hspace{10mm}		2*(h-2)\\
.\\
.\\
$2^{h-1}$	\hspace{5mm}	2*(h-(h-1))\\
$2^h$ 	\hspace{10mm}		2*(h-h)\\
\\
$h = \log n$\\
Average time will be\\
T(n) = $(2^{h-1} * 2 + 2^{h-2} * 2*2 + \ldots + 2*2*(h-1) + 1*2*h)/n$\\
T(n) = $(2/n)(1*2^{h-1} + 2*2^{h-2}+ \ldots + 2*(h-1) + 1*(h))$\\
T(n) = $(2/n) (1*h + 2*(h-1) + \ldots + 2^{h-2} * (h - (h-2)) + 2^(h-1) * (h-(h-1))$\\
T(n) = $(2*2^h/n) (h/2^h + (h-1)/2^{h-1} + \ldots + 2/2^2 + 1/2^1)$\\  
Since $2^h=n$\\
T(n) = $2*(h/2^h + (h-1)/2^{h-1} + \ldots + 2/2^2 + 1/2^1)$\\
T(n) = $2* \sum\limits_{j=1}^h j/2^j$
\\
Lets find this by small trick, for $x<1$\\
$ \sum\limits_{j=1}^{\infty} x^j = \frac{1}{1-x} $\\
Take derivative on both sides\\
$ \sum\limits_{j=1}^{\infty} j * x^{j-1} = \frac{1}{({1-x})^{2}}$\\
Multiply both sides with x and put x=1/2\\
$ \sum\limits_{j=1}^{\infty} j * (1/2)^{j} = \frac{1/2}{({1/2})^{2}}$\\
$ \sum\limits_{j=1}^{\infty} \frac{j}{2^j} = 2$\\
Since
$ \sum\limits_{j=1}^{\infty} \frac{j}{2^j} \leq \sum\limits_{j=1}^{h} \frac{j}{2^j} $\\
Therefore, substituting this value in our result\\
T(n) = 2*2=4=O(1)\\
\end{solution}
\fi

\end{itemize}

Assume that we wish to augment our 2-3 tree data structure so that
that each node $v$ maintains a pointer $v.succ$ to the successor of
$v$, so that queries for the successor of an element can be answered
in $O(1)$ time {\em worst-case}.

\begin{itemize}
\item[(c)] (6 points) Show that the 2-3 trees can be augmented while
  maintaining $v.succ$, such that the {\sc Insert} and {\sc Delete}
  operations can still be performed in $O(\log n)$ time.
\hint{Think of a linked list.}

 \ifnum\me<2
\begin{solution}
This problem can be solved using the linked list data structure. . The smallest(first) node's successor element can be stored in say top of linked list. The largest element successor link will currently point to NULL. Each leaf node will have a link to its successor. This way, we can find out successor of any element in time O(1). \\
\\
Now, whenever an element is inserted, we will need to make two or  update depending upon the inserted element. Suppose x is inserted into the 2-3 tree.\\
First of all, we will compare x with the smallest element by going to the left most of the tree, which will take O($\log n$) time. We can directly compare x with the largest element in O(1) time because the root will have the maximum value.\\
Intitally we will insert x normally and then\\
 There will be 3 cases\\
 
1. If x is smallest\\
In this case we will make new node successor's link to top and then assign top to new node. It will take O($\log n$) time because we have to find the smallest element first.\\
\\
2. if x is largest\\
In this case we will find the right most element first and we will add a link to it to point to the new node. Then we will add a link to the last node and point it to NULL. This procedure will also take time O($\log n$)\\
\\
3. If x has a predecessor as well as successor\\
In this case, we will have to find the position where x is inserted first. Then we will have to find the successor and predecessor of x which can be done in O($\log n$) time. Then we will have to make the following changes\\
(a). New node link now points to the successor of it.\\
(a). Make predecessor link now points to the new node\\
This whole procedure will also take O($\log n$) time
\\
\\
Now suppose we have to \textbf{delete an element x} from the 2-3 tree\\
There will again be 3 cases\\
1. If x is smallest i.e. it is at the top of the linked list\\
We will make it successor as top and delete the node x. Time taken will be O(1) in this case.\\
\\
2. If x is largest i.e. it is equal to the root of the tree\\
We will find its predecessor and make its link point to NULL and delete x. This will take time O($\log n$), since to find predecessor it takes time O($\log n$)\\
 \\
 3. If x has a predecessor as well as a successor\\
 We will find its predecessor and successor and then will change predecessor's link to point to successor. Then we will delete x. This will also take time O($\log n$) since finding predecessor and successor takes O($\log n$) time.\\
\\
\\
Hence for each new element insertion and deletion, these operations can be performed in O($\log n$) time and each node successor can be maintained.\\

\end{solution}
\fi

\end{itemize}







\end{document}
